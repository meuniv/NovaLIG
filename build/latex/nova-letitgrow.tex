%% Generated by Sphinx.
\def\sphinxdocclass{report}
\documentclass[letterpaper,10pt,english,openany,oneside]{sphinxmanual}
\ifdefined\pdfpxdimen
   \let\sphinxpxdimen\pdfpxdimen\else\newdimen\sphinxpxdimen
\fi \sphinxpxdimen=.75bp\relax

\PassOptionsToPackage{warn}{textcomp}
\usepackage[utf8]{inputenc}
\ifdefined\DeclareUnicodeCharacter
% support both utf8 and utf8x syntaxes
  \ifdefined\DeclareUnicodeCharacterAsOptional
    \def\sphinxDUC#1{\DeclareUnicodeCharacter{"#1}}
  \else
    \let\sphinxDUC\DeclareUnicodeCharacter
  \fi
  \sphinxDUC{00A0}{\nobreakspace}
  \sphinxDUC{2500}{\sphinxunichar{2500}}
  \sphinxDUC{2502}{\sphinxunichar{2502}}
  \sphinxDUC{2514}{\sphinxunichar{2514}}
  \sphinxDUC{251C}{\sphinxunichar{251C}}
  \sphinxDUC{2572}{\textbackslash}
\fi
\usepackage{cmap}
\usepackage[T1]{fontenc}
\usepackage{amsmath,amssymb,amstext}
\usepackage{babel}



\usepackage{times}
\expandafter\ifx\csname T@LGR\endcsname\relax
\else
% LGR was declared as font encoding
  \substitutefont{LGR}{\rmdefault}{cmr}
  \substitutefont{LGR}{\sfdefault}{cmss}
  \substitutefont{LGR}{\ttdefault}{cmtt}
\fi
\expandafter\ifx\csname T@X2\endcsname\relax
  \expandafter\ifx\csname T@T2A\endcsname\relax
  \else
  % T2A was declared as font encoding
    \substitutefont{T2A}{\rmdefault}{cmr}
    \substitutefont{T2A}{\sfdefault}{cmss}
    \substitutefont{T2A}{\ttdefault}{cmtt}
  \fi
\else
% X2 was declared as font encoding
  \substitutefont{X2}{\rmdefault}{cmr}
  \substitutefont{X2}{\sfdefault}{cmss}
  \substitutefont{X2}{\ttdefault}{cmtt}
\fi


\usepackage[Bjarne]{fncychap}
\usepackage{sphinx}

\fvset{fontsize=\small}
\usepackage{geometry}


% Include hyperref last.
\usepackage{hyperref}
% Fix anchor placement for figures with captions.
\usepackage{hypcap}% it must be loaded after hyperref.
% Set up styles of URL: it should be placed after hyperref.
\urlstyle{same}
\addto\captionsenglish{\renewcommand{\contentsname}{Contents:}}

\usepackage{sphinxmessages}
\setcounter{tocdepth}{1}



\title{Nova \sphinxhyphen{} Let it grow}
\date{Nov 28, 2020}
\release{}
\author{Vincent Meunier}
\newcommand{\sphinxlogo}{\vbox{}}
\renewcommand{\releasename}{}
\makeindex
\begin{document}

\pagestyle{empty}
\sphinxmaketitle
\pagestyle{plain}
\sphinxtableofcontents
\pagestyle{normal}
\phantomsection\label{\detokenize{index::doc}}



\chapter{Welcome to \sphinxstyleemphasis{Let it grow}}
\label{\detokenize{introduction:welcome-to-let-it-grow}}\label{\detokenize{introduction:introduction}}\label{\detokenize{introduction::doc}}
This module is designed to help you explore where your food comes from and how agriculture affects your life each day. Agriculture includes growing crops and raising animals to provide food and other products.

\begin{sphinxadmonition}{warning}{Warning:}
When completing this Award both the youth and involved adult leaders must obey all rules of \sphinxhref{https://www.scouting.org/health-and-safety/gss/}{Safe Scouting}. This includes (1) Completing Cyber Chip training prior to starting this activity and (2) \sphinxstylestrong{ALWAYS} involve at least 2 adults in all your communications with a leader, including online. If you send email to your counselor, always add the address of another adult leader or a parent/guardian. Never reply to a message sent by an adult leader unless another adult has been copied on the email. Report any issue to your parents/guardians!
\end{sphinxadmonition}


\section{Instructions}
\label{\detokenize{introduction:instructions}}\begin{enumerate}
\sphinxsetlistlabels{\arabic}{enumi}{enumii}{}{.}%
\item {} 
Identify a \sphinxstylestrong{Nova Counselor} either within your unit, district, or council.

\item {} 
This site provides you a platform for learning and you can easily follow all requirements using the navigation menu on the left.

\item {} 
Once you have identified a Counselor, you can start working on requirements.

\item {} 
The most important aspect in any scientific endeavor is to \sphinxstylestrong{properly document progress}. This will be done, here, using a google sheet as described in more details below.

\end{enumerate}


\section{Documenting your progress}
\label{\detokenize{introduction:documenting-your-progress}}\begin{enumerate}
\sphinxsetlistlabels{\arabic}{enumi}{enumii}{}{.}%
\item {} 
A template worksheet can be found \sphinxhref{https://docs.google.com/document/d/1tOlJcGxA8rKp7cc1t8yDhrckO1bbwJQjPA0HgETOAyI/edit?usp=sharing}{here}. This is a \sphinxstyleemphasis{Google document}. \sphinxstylestrong{You will not be able to modify it until you make your own copy as I will now describe for you.}

\item {} 
Once you have opened the file on google doc, go to \sphinxcode{\sphinxupquote{File}} \(\rightarrow\) \sphinxcode{\sphinxupquote{Make a Copy}}.

\item {} 
Save the file with the following name: \sphinxstyleemphasis{Nova\_lig\_FIRSTNAME\_LASTNAME}

\item {} 
You will use that file to enter your progress and share with your counselor.

\item {} \begin{description}
\item[{You can share your own copy of the worksheet with your counselor using the following procedure.}] \leavevmode\begin{enumerate}
\sphinxsetlistlabels{\alph}{enumii}{enumiii}{}{)}%
\item {} 
Click on the SHARE button on the top\sphinxhyphen{}right.

\item {} 
Click on “get link”.

\item {} 
Send the link to your counselor.

\end{enumerate}

\end{description}

\end{enumerate}

\begin{sphinxadmonition}{note}{Note:}
This document provides you a guide to complete the Nova award! All requirements are marked with the following symbol: \(\boxed{\mathbb{REQ}\Large \rightsquigarrow}\). In addition, a number of fun \sphinxstyleemphasis{Additional Challenges} are provided in boxes for your entertainment.
\end{sphinxadmonition}


\section{If you have any question}
\label{\detokenize{introduction:if-you-have-any-question}}
Contact your counselor or your scoutmaster! If you have questions about the program, contact Vincent Meunier  by \sphinxhref{mailto:vinmeunier@gmail.com}{email} (always make sure to copy another adult on all your communications!).


\section{Other Nova modules in this series}
\label{\detokenize{introduction:other-nova-modules-in-this-series}}
\begin{sphinxadmonition}{note}{More will be added, check regularly!}

\sphinxhref{https://novadtc.readthedocs.io}{\sphinxincludegraphics[scale=0.8]{{logo-dtc2_black}.png}}

\sphinxhref{https://novashoot.readthedocs.io}{\sphinxincludegraphics[scale=0.8]{{logo-shoot_black}.png}}

\sphinxhref{https://novalig.readthedocs.io}{\sphinxincludegraphics[scale=0.8]{{logo-lig_black}.png}}
\end{sphinxadmonition}


\chapter{Requirement \#1: Research and Reading}
\label{\detokenize{requirement1:requirement-1-research-and-reading}}\label{\detokenize{requirement1::doc}}\begin{description}
\item[{\(\boxed{\mathbb{REQ}\Large \rightsquigarrow}\) Choose A or B or C and complete all the requirements.}] \leavevmode\begin{enumerate}
\sphinxsetlistlabels{\Alph}{enumi}{enumii}{}{.}%
\item {} 
Watch about three hours total of shows or documentaries related to agriculture or farming. Then do the following:
\begin{enumerate}
\sphinxsetlistlabels{\arabic}{enumii}{enumiii}{(}{)}%
\item {} 
Make a list of at least five questions or ideas from the show(s) you watched.

\item {} 
Discuss two of the questions or ideas with your counselor.

\end{enumerate}

\begin{sphinxadmonition}{tip}{Tip:}
Some examples include—but are not limited to—shows found on PBS (“America’s Heartland,” “America Revealed,” “NOVA”), Discovery Channel, Science Channel, National Geographic Channel, History Channel, TED Talks (online videos), or “Good Eats” on the Food Network. You may choose to watch a live performance or movie at a planetarium or science museum instead of watching a media production. You may watch online productions with your counselor’s approval and under your parent’s supervision; appropriate websites include www.americasheartland.orga and www.neok12.com/Agriculture.htm
\end{sphinxadmonition}

\item {} 
Read (about three hours total) about anything related to agriculture or farming. Do the following.
\begin{enumerate}
\sphinxsetlistlabels{\arabic}{enumii}{enumiii}{(}{)}%
\item {} 
Make a list of at least five questions or ideas from each article

\item {} 
Discuss two of the questions or ideas with your counselor

\end{enumerate}

\begin{sphinxadmonition}{tip}{Tip:}
Books on these topics may be found at your local library. Examples of magazines include—but are not limited to—Odyssey, Kids Discover, National Geographic Kids, Highlights, and Owl.
\end{sphinxadmonition}

\item {} 
Do a combination of reading and watching (about three hours total). Then do the following:
\begin{enumerate}
\sphinxsetlistlabels{\arabic}{enumii}{enumiii}{(}{)}%
\item {} 
Make a list of at least two questions or ideas from each article or show.

\item {} 
Discuss two of the questions or ideas with your counselor.

\end{enumerate}

\begin{sphinxadmonition}{tip}{Tip:}
Examples of magazines include—but are not limited to—Odyssey, Popular Mechanics, Popular Science, Science Illustrated, Discover, Air \& Space, Popular Astronomy, Astronomy, Science News, Sky \& Telescope, Natural History, Robot, Servo, Nuts and Volts, and Scientific American.
\end{sphinxadmonition}

\end{enumerate}

\end{description}

\begin{sphinxadmonition}{note}{Additional Challenge}
\begin{enumerate}
\sphinxsetlistlabels{\arabic}{enumi}{enumii}{}{.}%
\item {} 
\sphinxstylestrong{What crop produces the most cash for American farmers?}
\begin{enumerate}
\sphinxsetlistlabels{\alph}{enumii}{enumiii}{(}{)}%
\item {} 
Corn;

\item {} 
Wheat;

\item {} 
Apple;

\item {} 
Potatoes

\end{enumerate}

\item {} 
\sphinxstylestrong{Which of these units is the most common way to measure the size of a farm in the U.S.?}
\begin{enumerate}
\sphinxsetlistlabels{\alph}{enumii}{enumiii}{(}{)}%
\item {} 
Square miles;

\item {} 
Bushels;

\item {} 
Acres;

\item {} 
Cubic feet

\end{enumerate}

\item {} 
\sphinxstylestrong{Which type of livestock produces the most money for U.S. farmers?}
\begin{enumerate}
\sphinxsetlistlabels{\alph}{enumii}{enumiii}{(}{)}%
\item {} 
Cattle;

\item {} 
Pigs;

\item {} 
Chickens;

\item {} 
Goats

\end{enumerate}

\end{enumerate}

Note: These questions are inspired by a trivia found on \sphinxhref{https://play.howstuffworks.com/quiz/can-you-answer-these-agriculture-questions-a-farmer-should-know}{how stuff works}.
\end{sphinxadmonition}

\begin{sphinxadmonition}{attention}{Attention:}
Once you have completed this requirement, make sure you document it in your worksheet!
\end{sphinxadmonition}


\chapter{Requirement \#2: Merit Badge}
\label{\detokenize{requirement2:requirement-2-merit-badge}}\label{\detokenize{requirement2::doc}}
\(\boxed{\mathbb{REQ}\Large \rightsquigarrow}\) Complete ONE merit badge from the following list. Choose one that you have not already used toward another Nova award.
After completion, discuss with your counselor how the merit badge you earned uses agriculture.
\begin{itemize}
\item {} 
Animal Science

\item {} 
Cooking

\item {} 
Farm Mechanics

\item {} 
Fish and Wildlife Management

\item {} 
Fishing

\item {} 
Forestry

\item {} 
Gardening

\item {} 
Insect Study

\item {} 
Mammal Study

\item {} 
Nature

\item {} 
Plant Science

\item {} 
Soil and Water Conservation

\end{itemize}

\begin{figure}[htbp]
\centering

\noindent\sphinxincludegraphics[width=700\sphinxpxdimen]{{meritbadges}.png}
\end{figure}

\begin{sphinxadmonition}{attention}{Attention:}
Once you have completed this requirement, make sure you document it in your worksheet!
\end{sphinxadmonition}


\chapter{Requirement \#3: Be a farmer}
\label{\detokenize{requirement3:requirement-3-be-a-farmer}}\label{\detokenize{requirement3::doc}}
\(\boxed{\mathbb{REQ}\Large \rightsquigarrow}\) Act like a farmer! Think about crops or animals that are found on a farm, and think about the different kinds of farms. Then choose TWO from A or B or C
\begin{enumerate}
\sphinxsetlistlabels{\Alph}{enumi}{enumii}{}{.}%
\item {} 
\sphinxstylestrong{With your counselor, choose two of the following topics related to food production or processing, and investigate them. Discuss your findings with your counselor}
\begin{enumerate}
\sphinxsetlistlabels{\arabic}{enumii}{enumiii}{(}{)}%
\item {} 
Where did the food you ate for dinner last night come from? Pick one food item and learn more about each of its ingredients. Where were those ingredients grown, and how did the food item get to your table?

\item {} 
What kind of equipment is used on a farm?

\item {} 
How were food plants invented? Where do most food plants come from?

\item {} 
How and why are scientists working to develop plants that don’t need as much water?

\item {} 
If a big disaster wiped out a lot of food plants, how would we get more of them? How do seed banks work?

\end{enumerate}

\item {} 
\sphinxstylestrong{Define and learn about two of the following, and discuss with your counselor.}
\begin{enumerate}
\sphinxsetlistlabels{\arabic}{enumii}{enumiii}{(}{)}%
\item {} 
Farming practice categories (conventional, sustainable, till, low\sphinxhyphen{}till, and no\sphinxhyphen{}till)

\item {} 
Conventional, organic, and biotech farming (compare and contrast)

\item {} 
Effects of weather on farming

\item {} 
Converting biomass into energy

\item {} 
STEM careers in agriculture (food science, plant science, farming, agricultural engineering)

\end{enumerate}

\item {} 
\sphinxstylestrong{Do an “agriscience” experiment and discuss the results with your counselor. Examples of experiments include — but are not limited to — the following:}
\begin{enumerate}
\sphinxsetlistlabels{\arabic}{enumii}{enumiii}{(}{)}%
\item {} 
Grow different types of seeds and compare the seedling plants. Use fast\sphinxhyphen{}growing seeds such as carrots, castor beans, lima beans, onions, radishes, soybeans, or tomatoes.

\item {} 
Select and study a specific growing variable such as the type of liquid used to water a seed, the type of light, the growing temperature, or the soil type.

\begin{sphinxadmonition}{tip}{Tip:}
Examples of growing studies can be found at www.agclassroom.org/teen/science/idealab.htm and www.sciencekids.co.nz/projects/plants.html
\end{sphinxadmonition}

\item {} 
People often think of microorganisms as germs, but many of the ones found in soil are good for agriculture. How can plants grow in soil if no microorganisms are present?

\begin{sphinxadmonition}{tip}{Tip:}
Search the internet — with your parent’s permission — and find an experiment that can be done to test the effect of microorganisms. Then perform the experiment.
\end{sphinxadmonition}

\end{enumerate}

\end{enumerate}

\begin{figure}[htbp]
\centering
\capstart

\noindent\sphinxincludegraphics[width=600\sphinxpxdimen]{{1_3LYHYoXdEpguaL3leuA1tA}.jpeg}
\caption{Image obtained from Fast Company (click on image for reference)}\label{\detokenize{requirement3:id1}}\end{figure}

\begin{sphinxadmonition}{attention}{Attention:}
Once you have completed this requirement, make sure you document it in your worksheet!
\end{sphinxadmonition}


\chapter{Requirement \#4: Visit}
\label{\detokenize{requirement4:requirement-4-visit}}\label{\detokenize{requirement4::doc}}
\(\boxed{\mathbb{REQ}\Large \rightsquigarrow}\) Visit a farm, botanical garden, grocery store, or any other location where farm produce can be found.

\begin{sphinxadmonition}{tip}{Tip:}
You can complete this requirement using a \sphinxstyleemphasis{virtual} visit!
\end{sphinxadmonition}
\begin{enumerate}
\sphinxsetlistlabels{\Alph}{enumi}{enumii}{}{.}%
\item {} 
During your visit, talk with someone in charge about how the plants are grown or animals are raised, and how the food is processed.

\item {} 
Discuss with your counselor the food science involved at the place you visited.

\end{enumerate}

\begin{sphinxadmonition}{note}{Fun farming facts}
\begin{itemize}
\item {} 
The average dairy cow produces 46,000 glasses of milk per year! There are six main breeds of dairy cows with the most common being the traditional black and white Holsteins. No two cows are identical, but the average cow weighs 1,200 pounds (or 544 kilos).

\item {} 
While cattle are ruminants and grazers, sheep are as well. By contrast, the differences between sheep and goats is that goats evolved to eat off the ground over time. Goats are browsers, meaning they tend to eat branches, leaves, shrubs, vines, and the like, primarily.

\item {} 
Tractors were invented in the 1890s to pull plows through fields. By the 1920s, the all purpose modern tractor had been developed. With different attachments, tractors can be used for plowing, cultivating, mowing, planting, harvesting, and moving soil and heavy equipment. Before tractors, it wasn’t uncommon to use horses or oxen on the farm.

\item {} 
Horses are the original man’s best friend and are estimated to have been domesticated since 6000 B.C. They have the largest eyes of any land mammal, and their teeth take up more room in their head than their brain. Although the average domesticated horse has a lifespan of 25 years, the oldest horse ever recorded was “Old Billy” at 62 years old.

\item {} 
Whether food crops are cultivated mechanically with tractors, with horses, or by hand, another fun fact is there is over 1,600 varieties of bananas! There are hundreds of thousands of varieties of all different kinds of edible plants.

\item {} 
Fun facts about chickens? The average chicken lays an egg a day. There are more chickens than any other bird species in the world, with 25 billion chickens around the globe.

\end{itemize}

These fun facts about farming were copied from \sphinxhref{https://www.agdaily.com/lifestyle/farm-babe-impress-your-friends-with-off-beat-farm-trivia/}{agdaily.com}. Check out the website for more facts!
\end{sphinxadmonition}

\begin{figure}[htbp]
\centering
\capstart

\noindent\sphinxincludegraphics[width=600\sphinxpxdimen]{{EUFarm}.jpg}
\caption{Farmers could protect the environment and cut down on fertiliser use with swarms of drones. Image from robohub.org. Click on image for full credit.}\label{\detokenize{requirement4:id1}}\end{figure}

\begin{sphinxadmonition}{attention}{Attention:}
Once you have completed this requirement, make sure you document it in your worksheet!
\end{sphinxadmonition}


\chapter{Requirement \#5: Farming@Life}
\label{\detokenize{requirement5:requirement-5-farming-life}}\label{\detokenize{requirement5::doc}}
\(\boxed{\mathbb{REQ}\Large \rightsquigarrow}\) Discuss with your counselor how farming affects your everyday life.

\begin{sphinxadmonition}{note}{Additional Challenge}
\begin{enumerate}
\sphinxsetlistlabels{\arabic}{enumi}{enumii}{(}{)}%
\item {} 
\sphinxstylestrong{Why are honey bees so important to plants worldwide?}
\begin{enumerate}
\sphinxsetlistlabels{\alph}{enumii}{enumiii}{(}{)}%
\item {} 
They help plants and flowers grow;

\item {} 
They keep other insects away from plants;

\item {} 
They rub off nutrients from one plant onto another;

\item {} 
They pollinate flowers and plants and help them reproduce.

\end{enumerate}

\item {} 
\sphinxstylestrong{What are the “female” parts of a plant or flower called?}
\begin{enumerate}
\sphinxsetlistlabels{\alph}{enumii}{enumiii}{(}{)}%
\item {} 
Pistil;

\item {} 
Ovary;

\item {} 
Stamen;

\item {} 
Stigma

\end{enumerate}

\item {} 
\sphinxstylestrong{How can you tell how old a tree is?}
\begin{enumerate}
\sphinxsetlistlabels{\alph}{enumii}{enumiii}{(}{)}%
\item {} 
By counting the number of rings in its trunk;

\item {} 
By counting its leaves;

\item {} 
By measuring its height;

\item {} 
By counting its branches

\end{enumerate}

\end{enumerate}

Note: These questions are inspired by a trivia found on \sphinxhref{https://play.howstuffworks.com/quiz/the-plant-quiz}{how stuff works}.
\end{sphinxadmonition}

\begin{figure}[htbp]
\centering
\capstart

\noindent\sphinxincludegraphics[width=700\sphinxpxdimen]{{dendro_1}.jpg}
\caption{Dendochronology: the science of tree\sphinxhyphen{}ring dating. Image taken from \sphinxhref{http://www.pbs.org/time-team/experience-archaeology/dendrochronology/}{pbs.org}. Visit the site for more information.}\label{\detokenize{requirement5:id1}}\end{figure}

\begin{sphinxadmonition}{attention}{Attention:}
Once you have completed this requirement, make sure you document it in your worksheet!
\end{sphinxadmonition}


\chapter{About the author}
\label{\detokenize{contact:about-the-author}}\label{\detokenize{contact::doc}}
These pages were written by Vincent Meunier, the Chair of the STEM committee of \sphinxhref{https://www.trcscouting.org}{Twin Rivers Council} in New York State.

Vincent Meunier is a Professor of physics at Rensselaer Polytechnic Institute. If you have any question, feel free to contact him by \sphinxhref{mailto:vinmeunier@gmail.com}{email}.

\begin{sphinxadmonition}{note}{Note:}
Most of the material used here was obtained from a number of external scouting sources, including \sphinxhref{https://www.scouting.org/wp-content/uploads/2018/11/Designed-to-Crunch-Nova-2018Nov26.pdf}{scouting.org}
\end{sphinxadmonition}

\begin{figure}[htbp]
\centering
\capstart

\noindent\sphinxincludegraphics[width=600\sphinxpxdimen]{{6f03ea444eb73c1afd5ddf05b1e6b62be276fa12}.jpeg}
\caption{Agriculture is no exception to digital revolution, going beyond a simple adoption of Information and Communication Technologies. Image obtained from edx.com}\label{\detokenize{index:id1}}\end{figure}



\renewcommand{\indexname}{Index}
\printindex
\end{document}